% Test for various startalign features without equation numbering
\starttext
% Simple Alignment (no need to specify the number of columns)

\startformula \startalign
  \NC a_1 x + b_1 y \NC = c_1 \NR
  \NC a_2 x + b_2 y \NC = c_2 \NR
\stopalign \stopformula

% Change the number of columns
\startformula \startalign[n=3]
  \NC a_1 x + b_1 y \NC = c_1 \NC =  d_1 u + e_1 v \NR
  \NC a_2 x + b_2 y \NC = c_2 + c_3 \NC =  d_2 u + e_2 v + f_1 w\NR
\stopalign \stopformula

% Change the alignment of columns
\startformula \startalign[n=4, align={right,middle,middle,left}]
  \NC a_1 x + b_1 y \NC = \NC c_1 \NC =  d_1 u + e_1 v \NR
  \NC a_2 x + b_2 y \NC = \NC c_2 + c_3 \NC =  d_2 u + e_2 v + f_1 w\NR
\stopalign \stopformula


% Separated alignments
\startformula \startalign[m=2]
  \NC a_1 x + b_1 y \NC = c_1 \NC d_1 u + e_1 v \NC = f_1 \NR
  \NC a_2 x + b_2 y \NC = c_2 \NC d_2 u + e_2 v \NC = f_2 \NR
\stopalign \stopformula

% Change distance
\startformula \startalign[m=2,distance=5em]
  \NC a_1 x + b_1 y \NC = c_1 \NC d_1 u + e_1 v \NC = f_1 \NR
  \NC a_2 x + b_2 y \NC = c_2 \NC d_2 u + e_2 v \NC = f_2 \NR
\stopalign \stopformula

% Combination of some options
\startformula \startalign[m=2,distance=8em,n=5,
  align={right,middle,middle,middle,left}]
  \NC 0 \NC < \NC 2x + 5y \NC < \NC 10
  \NC 4 \NC < \NC 3x +  y \NC < \NC 9  \NR
  \NC 3 \NC < \NC 2y + 3z \NC < \NC 15
  \NC 10\NC < \NC 8y + 5z \NC < \NC 20 \NR
\stopalign \stopformula

% startformulas
\startformulas
  \startformula \startalign
    \NC a_1 x + b_1 y \NC = c_1 \NR
    \NC a_2 x + b_2 y \NC = c_2 \NR
  \stopalign \stopformula
  \startformula \startalign
    \NC d_1 u + e_1 v \NC = f_1 \NR
    \NC d_2 u + e_2 v \NC = f_2 \NR
  \stopalign \stopformula
\stopformulas

% Different number of lines in each block
\startformulas
  \startformula \startalign
    \NC 2x + 3 \NC = 7 \NR
    \NC 2x \NC = 4 \NR
    \NC x \NC = 2 \NR
  \stopalign \stopformula
  \startformula \startalign
    \NC x^2 + 2x \NC = 3 \NR
    \NC x^2 + (3-1)x - 3 \NC = 0 \NR
    \NC x(x + 3) -1(x + 3) \NC = 0 \NR
    \NC (x + 3)(x - 1) \NC = 0 \NR
    \NC x \NC = -3 \text{ or } 1 \NR
  \stopalign \stopformula
\stopformulas


\stoptext
