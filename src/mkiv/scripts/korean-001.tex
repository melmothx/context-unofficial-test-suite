% Samples by Jeong Dalyoung.

\mainlanguage[kr]

\definefontfeature[korean][script=hang,language=kor,mode=node,analyze=yes]

\starttypescript [serif] [myungjo]
    \definefontsynonym[Serif]    [adobemyungjostd-medium*korean]
    \definefontsynonym[SerifBold][adobemyungjostd-medium*korean]
\stoptypescript

\starttypescript[myungjo]
    \definetypeface [myungjo] [rm] [serif] [myungjo] [default] [features=korean]
    \definetypeface [myungjo] [mm] [math]  [modern]  [default]
\stoptypescript

\setupbodyfont[myungjo]

\setuplayout[width=middle,height=middle,backspace=1cm,topspace=1cm,footer=0pt]

\starttext

% "ㄱ" \lchexnumber{`ㄱ}\quad
% "ㄴ" \lchexnumber{`ㄴ}\quad
% "ㄷ" \lchexnumber{`ㄷ}\quad
% "ㄹ" \lchexnumber{`ㄹ}\quad
% "ㅁ" \lchexnumber{`ㅁ}\quad
% "ㅂ" \lchexnumber{`ㅂ}\quad
% "ㅅ" \lchexnumber{`ㅅ}\quad
% "ㅇ" \lchexnumber{`ㅇ}\quad
% "ㅈ" \lchexnumber{`ㅈ}\quad
% "ㅊ" \lchexnumber{`ㅊ}\quad
% "ㅋ" \lchexnumber{`ㅋ}\quad
% "ㅌ" \lchexnumber{`ㅌ}\quad
% "ㅍ" \lchexnumber{`ㅍ}\quad
% "ㅎ" \lchexnumber{`ㅎ}
% \blank[big]
% \enabletrackers[sorters.methods,sorters.tests]
% \placeregister[index][n=5]
% \index{첫 6개}
% \index{첫6개}
% \index{첫개6}
% \blank[4*big]

\chapter{머릿말}

\index {조합론}은 \index {이산적 구조}를 가진
문제들을 다루는 \index {수학}의 한 분야이다. \index
{연구}의 영역은 어떤 물건들을 미리 정해진 \index
{조건}에 따라 \index {선택하고 배열}하는 일들,
어떤 점들과 그 점들의 \index {연결관계}로
이루어진 \index {그래프}(\index {graph})라고 불리는
어떤 \index {구조}에 대한 연구, 그리고 \index
{특정}한 규칙에 따른 \index {실험계획}의 \index
{디자인} 등을 포함한다. 조합론적 문제들과 그
응용들은 수학의 여러 분야에서만 발견되는 것이
아니라 \index {공학}, \index {컴퓨터이론}, \index {OR},
\index {경영과학}, 그리고 \index {생명과학}의
분야와 같은 다른 영역에서도 에서도 발견되고
있다. 컴퓨터가 문제들의 이산적 구성을 요구하기
때문에 조합론적 기법이 공학자나 \index
{응용과학자}들에게, 예를 들면, \index {조선사}의
선박 \index {건조일정}을 계획하는 일부터
생명과학의 \index {인간 유전자} 연구에
이르기까지, \index {필수적}이고 \index {강력한
도구}가 되었다.

특별한 경우에 얼마나 많은 배열이 있는가를 찾는
문제들인, 경우의 수를 세는 문제들은 조합론의
기본적인 문제들 중의 하나이다. \index {Counting}은
\index {사회과학}에서 \index {의사결정} 기관의
(\index {주주총회}, \index {의회}, 그리고 \index {UN
안전보장이사회}와 같은) 참여자들의 \index
{영향력}을 \index {측정}하는 \index {샤플리-수빅}
(\index {Shapley-Shubig}) \index {파워 지수}를
계산하는데 이용되기도 하였다. \index
{화학}에서는 \index {케일리}(\index {Cayley})가 \index
{포화탄화수소}의 \index {이성질체}의 개수를
세기위해 그래프를 이용하였다; 한편, \index
{해라리}(\index {Harary})와 \index {리드}(\index {Read})는
\index {벤젠 링}으로 부터 만들어지는 어떤 \index
{유기화합물}을 공유된 변을 따라 연결된 \index
{육각형}의 \index {구조물}로 나타냄으로써 그
개수를 셀 수 있었다. 유전공학에서는 네 개의
\index {기본 핵산}들로 구성된 모든 가능한 \index
{DNA} \index {연결구조}를 셈으로써 \index
{과학자}들은 놀랄만한 큰 수에 \index {도달}하게
되었고, 그 결과로 유전자를 구성할 수 있는
엄청나게 많은 \index {가능성}을 \index {이해}할 수
있게 되었다. Counting은 \index {RNA}의 제일, 제이의
구조를 연구하는데 사용되기도 하였다.

이 책은 \index {상급} \index {고등학생}들, \index
{대학} \index {초년생}들, 그리고 \index
{선생님}들에게 기본적인 조합론적 기법을
소개하기위해 쓰여졌다. 또한 이 책이 수학을
즐기는 사람들과 의욕있는 \index {퍼즐가}들에게
\index {흥미로운 책}이 될 것이라 믿는다.

이 책의 \index {다양한} \index {문제}들과 응용들은
counting의 \index {능력}을 기르는데 유용한 것만이
아니라 일반적인 \index {문제해결} (\index
{Problem-solving})의 기본 능력과 기법을 \index
{연마}하는 \index {풍부한} \index {자료}가 될
것이다. 이 책의 많은 문제들이 \index {흔한
경우}들을 \index {가급적} 피하고 있어서, 그
결과로 독자들이 그 문제들을 해결하기위해 \index
{열심히} \index {생각}하도록 요구하고 있다.
실제로, \index {부지런한} \index {독자}는 어떤 \index
{특정한} 문제를 푸는데 한 가지보다 많은 방법을
발견하곤 하는데, 이것은 \index {진정} 문제해결에
있어 \index {중요한} \index {깨달음}이다. 따라서 이
\index {책자}는 \index {학생}들이 문제해결의 \index
{발견적 학습법}과 \index {사고 기법} 배우는 이른
\index {출발}을 할 수 있도록 돕고있다.

첫 두 장은 두 가지 \index {기본 원리}들, 즉, \index
{합의 원리}와 \index {곱의 원리}를 다루고 있다. 이
두 가지 원리는 Counting에 \index {일상적}으로
사용되는데, 자신이 수학을 공부하는 사람이
아니라고 생각하는 사람들에 의해서도 \index
{사용되고} 있다. \index {그렇지만} 이 두 원리가
\index {가끔} \index {잘못 이해}되고 \index {잘못
사용}되기도 한다. \index {1장}과 \index {2장}에서 이
\index {원리}들이 \index {적용}될 수 있는 \index
{조건}들을 \index {명확하게} 함으로써 이런 \index
{오류}를 피할 수 있게 도와준다. \index {3장}은
\index {부분집합}과 어떤 물건들의 배열을 통해
\index {조합}과 \index {순열}의 \index {개념}을
설명하고 있으며, \index {4장}에서는 이미 배운
개념들의 \index {다양한} 응용문제들을 \index
{제공}하고 있다.

\index {언뜻보기}에 \index {복잡하게} 보이는 많은
counting 문제들이 단순히 "\index {관점의 변화}"를
통해 해결될 수 있다. 5장에서는 이런 \index
{맥락}에서, 중요한 원리, 즉 \index {일대일 대응
원리}를 소개하고 있다; 한편 6장에서는 매우 \index
{유용한 관점}을 소개하고 있는데 그것은 많은
\index {세는 문제}들을 \index {상자}에 \index {공}을
\index {배분}하는 문제들로 바꾸어 생각할 수
있다는 것이다. 그 다음에 나오는 세 장들에서는
많은 응용문제들과 \index {변형}들을 통해 \index
{일대일 대응원리}와 배분 개념에 익숙해지게
하고 있다.

제 3 장에서 $n\choose r$ 혹은 $C_r^n$로 표현되는 많은
수들을 소개하였다. 마지막 세 장들은 이 수들을
\index {이항전개식}과 \index {파스칼의 삼각형}을
통해 더 다루고 있다. 많은 \index {유용한} \index
{항등식}들이 \index {증명}되었고, 또 이 \index
{항등식}들이 나타나는 의외의 문제들도
제시되었다.

\index {마지막}으로, 제 \index {13 장}에 앞에서 배운
하나 혹은 더 많은 개념들의 응용으로 생각할 수
있는 \index {흥미로운} 문제들을 모아 놓음으로써
이 책을 \index {마무리}하고 있다. 이 책에 (C)로
표시된 문제들은 \index {케임브리지 대학}의 \index
{Local} \index {Examinations} \index {Syndicate}의 \index
{허락}하에, 그리고 (\index {AIME})로 표시된
문제들은 \index {American} \index {Invitational} \index
{Mathematics} Examination의 허락하에 소개되었다. 이
문제들을 이 책에 소개할 수 있도록 허락해준 두
기관에 감사의 뜻을 표한다.

이 책은 \index {싱가포르} \index {수학회}의 \index
{잡지} 중, \index {Matematical Medley}에 처음으로 실린
\index {counting}에 대한 \index {일련의 글}들 중 \index
{첫 6개}의 글에 \index {기초}를 두고 있다. 원 \index
{시리즈} \index {집필}시 첫 저자를 많이 도와준
\index {Tan Ban Pin}에게 감사드린다. 또한 \index
{초안}을 읽어주고 문제들을 \index {확인}해준
우리의 \index {동료들}인 \index {Dong Fengming}, \index
{Lee Tuo Yeong}, 그리고 \index {Toh Tin Lam}에게도
감사드린다. - 이 책의 어떤 실수도 저자들에게
\index {책임}이 있다.


이와 같은 counting에 대한 소개에 \index {흥미}를
느낀 사람들과 이 \index {주제}에 대해 더 많은
것을 알기를 원하는 사람들을 위해 더 \index {깊은}
내용을 다루고 있는 책들의 \index {추천} \index
{리스트}를 이 책의 마지막 부분에 \index
{제시}하였다.

\index {탄광}에서 \index {석탄}을 \index {캐다가} \index
{진폐증}에 걸린 사람들이 많이 있는데 그 \index
{수효}는 정확하게 \index {파악되지} 않고 있다.
\index {쌍수}를 들어 \index {환영할} 일이지만 \index
{깜빡}하다가는 \index {큰일}이 날 수도 있다. \index
{땅}과 \index {바다}에 가득하다는 것은 무슨
의미가 있을까? 사람이 \index {떡}과 \index
{빵}만으로는 살 수가 없다. \index {하나님의
말씀}으로 살아야 한다. 우리 \index {동네}에 \index
{카센터}가 \index {유독} 많은 것은 \index {어떤
이유}인지 모르겠다. \index {큰바위얼굴}은 어렸을
때 감명깊게 \index {읽은 글}이다. \index {푸른하늘}
\index {은하수}도 어렸을 때 즐겨 \index {불렀던}
\index {노래}다.

\placefigure{test}{\framed{그림}}

\stoptext
